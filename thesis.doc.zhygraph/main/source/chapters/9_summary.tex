Láthattuk, hogyan épül fel egy grafikai megjelenítés orientált keretrendszer, ami egy szkript interfészen keresztül teszi elérhetővé a funkcióit. Ahogy a más, hasonló jellegű szoftverekkel történő összehasonlításkor is kiderült, egy ilyen keretrendszernek számtalan bővítési lehetősége van.

Ugyan ebben az esetben kifejezetten a vizuális funkciókra fektettük a hansúlyt, a rendszer tovább bővíthető fizikai szimulációt megvalósító motorral, hangfájlok feldolgozását végző alrendszerrel vagy akár mesterséges intelligencia funkciókkal is.

A tartalom nyílvántartó rendszer általános megvalósítása lehetővé teszi, hogy egyszerűen hozzunk létre új tartalom típusokat, a szkript felület pedig szerkezeténél fogva bövíthető, hiszen a funkciói csupán a keretrendszer belső működését tükrözik.

A funkciók bővítésével természetes módon nőhet az alkalmazási területek skálája is, az említett komponensek hozzáadásával akár egy teljesértékű videójáték motorrá lenne fejleszthető a rendszer...
