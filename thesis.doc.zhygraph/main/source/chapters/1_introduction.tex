\Chapter{Bevezetés}

Az információs technológia fontos szereplői a nagy teljesítményű grafikus processzorok. Felhasználási területeik egy igen széles skálát képeznek, magába foglalva a fotórealisztikus kép és videó alkotást, a \textit{közel} fotórealisztikus, \textit{valósidejű} megjelenítést, a grafikai és fizikai szimulációkat vagy akár a teljesen általános célú, de nagy hatékonysággal párhuzamosítható számítási problémákat is. 

Általánosan igaz, hogy egy nagy teljesítményű hardver teljes kihasználtságához elengedhetetlen a hardver működésének precíz ismerete, és valamilyen hardver-közeli programozási nyelv alkalmazása. Ahhoz azonban, hogy az ilyen nyelvekhez kellő ismereteket szerezzünk, általában sokkal több idő lenne szükséges, mint amennyi rendelkezésünkre áll egy adott alkalmazás lefejlesztéséhez. Továbbá, minél közelebb kerülünk a célhardver sajátságaihoz, annál kevésbé számíthatunk a futási környezet, vagy az operációs rendszer támogatására. Ez jellemzően azt eredményezi, hogy sokkal könnyebben jönnek létre szoftveres hibák, illetve sokkal több idő telik el a hibák javításával natív környezetben - mint a C/C++ nyelvek - a virtuális nyelvekhez - mint a Java, vagy a C\# - képest \cite{morebugsincpp}.

A problémát tehát, az jelenti, hogy a nagy számítási és memória igényű alkalmazásokat általában célszerű natív környezetben fejleszteni, a teljesítmény igények kielégítéséhez. Erre azonban nem minden esetben van lehetőség, bizonyos esetekben a megfelelő ismeretek hiánya, más esetben a fejlesztési és szoftvertámogatási időkeretek szűkössége miatt.

A továbbiakban egy olyan keretrendszert valósítunk meg, aminek segítségével nagy teljesítményű, grafikus alkalmazások hozhatóak létre, közvetlen technológiai ismeretek nélkül. Ez úgy lehetséges, hogy a bonyolult geometriai megjelennítés részleteit a natív C++ környezetben alakítjuk ki, míg az alkalmazás logika konkrét megvalósítása a felhasználó feladata lesz, egy egyszerűen használható szkript felület segítségével.

A legfőbb elvárás a keretrendszerrel szemben, hogy a kódbázis megváltoztatása nélkül, kizárólag adat- vagy szkriptfájlok létrehozásával, tetszőleges viselkedésű, illetve tetszőleges geometriai objektumokat, hatékonyan megjelenítő alkalmazást fejleszthessünk segítségével.

A következőkben bemutatjuk, hogy milyen hasonló rendszerek állnak rendelkezésünkre a probléma megoldásához, valamint, hogy miben hasonlít illetve miben tér el ezektől az új megoldás. Láthatjuk majd a keretrendszer részletes, architektúrális leírását, a rendszer mögött húzódó elméleti megfontolásokat valamint a felhasználóknak szóló, alkalmazás fejlesztést segítő leírást is.
