\pagestyle{empty} %a c�mlapon ne legyen semmi=empty, azaz nincs fejl�c �s l�bl�c

%A f�iskola logoja
{\large
\begin{center}
\vglue 1truecm
\textbf{\huge\textsc{Szakdolgozat}}\\
\vglue 1truecm
\epsfig{file=data/cover/ME_logo.eps, width=4.8truecm, height=4truecm}\\
\textbf{\textsc{Miskolci Egyetem}}
\end{center}}

\vglue 1.5truecm %f�gg�leges helykihagy�s

%A szakdolgozat c�me, ak�r t�bb sorban is
{\LARGE
\begin{center}
\textbf{Itt jelenik meg a szakdolgozat c�me, ak�r t�bb sorban is}
\end{center}}

\vspace*{2.5truecm}
%A hallgat� neve, �vfolyam, szak(ok), a konzulens(ek) neve
{\large
\begin{center}
\begin{tabular}{c}
\textbf{K�sz�tette:}\\
Ide ker�l a hallgat� neve\\
�vfolyam. szak-szak
\end{tabular}
\end{center}
\begin{center}
\begin{tabular}{c}
\textbf{T�mavezet�:}\\
Egyik konzulens neve\\
M�sik konzulens neve\ldots
\end{tabular}
\end{center}}
\vfill
%Keltez�s: Hely �s �v
{\large
\begin{center}
\textbf{\textsc{Miskolc, 2016}}
\end{center}}

\newpage
